\documentclass{article}

\usepackage{amsmath}
\usepackage{amssymb}
\usepackage{amsthm}
\usepackage[hidelinks]{hyperref}
\usepackage{xcolor}
\usepackage{graphicx}

\usepackage{geometry}
 \geometry{
 a4paper,
 total={170mm,257mm},
 left=20mm,
 top=20mm,
 }
\usepackage{blindtext}
\usepackage{lastpage}
\usepackage{fancyhdr}
\pagestyle{fancy}
\fancyhf{} 
\fancyfoot[R]{Page \thepage}

\newtheorem{definition}{Definition}
\newtheorem{theorem}{Theorem}
\newtheorem{lemma}{Lemma}
\newtheorem{proposition}{Proposition}
\newtheorem{example}{Example}
\newtheorem{corollary}{Corollary}

\begin{document}

Basic properties of complex numbers $z=x+iy$ are not mentioned here. Based on the definition of the absolute value (norm) of the complex number: $|z|=(x^2+y^2)^{1/2}$, two interesting properties are listed below:

\begin{enumerate}
\item
Triangle inequality: $|z_1+z_2|\leq|z_1|+|z_2|$

\item
$|z_1-z_2|\geq||z_1|-|z_2||$
\end{enumerate}
The first property can be verified by thinking of the complex number as a vector in $\mathbb{R}^2$ with the real part and complex part being respectively the first and second coordinates. The second property can be proved as follows:

\begin{proof}
If we can show both $|z_1-z_2|\geq|z_1|-|z_2|$ and $|z_1-z_2|\geq|z_2|-|z_1|$, then we are done. From the triangle inequality, we have:

\begin{equation*}
\begin{aligned}
&|z_1|=|z_2+z_1-z_2|\leq|z_2|+|z_1-z_2|\Rightarrow |z_1|-|z_2|\leq|z_1-z_2|\\
&|z_2|=|z_1+z_2-z_1|\leq|z_1|+|z_2-z_1|\Rightarrow |z_2|-|z_1|\leq|z_2-z_1|=|z_1-z_2|
\end{aligned}
\end{equation*}
\end{proof}

Roughly speaking, for a complex function $f:\mathbb{C}\rightarrow\mathbb{C}$, if $\frac{f(z+h)-f(z)}{h}$ converges as $h\rightarrow0$, then we say $f$ is holomorphic. Note that $h$ can go to $0$ from different directions, i.e., from the real axis, the complex axis, or many more. Being a holomorphic function in an open set $\Omega$ enables a lot of miracle to happen, such as:

\begin{enumerate}
\item
Contour integration: for closed paths, the contour integration of $f$ equals 0, independent of the parametrization.

\item
Regularity: $f$ is differentiable infinitely many times, and it has a convergent power series (Taylor series expansion).

\item
Analytic continuation: if $f$ and $g$ agree on a (possibly tiny) open subset of $\Omega$, then they agree on all $\Omega$.
\end{enumerate}

Note that the complex space is complete, which implies that a sequence converges if and only if it is a Cauchy sequence.

We recall several basic properties of a set here.

\begin{definition}\textcolor{red}{(Interior Point)}

For a set $\Omega$, $z\in\Omega$ is called the interior point if there exists $r>0$ such that the disk $D_r(z):={w\in\mathbb{C}:|z-w|<r}$ is contained in $\Omega$.
\end{definition}

\begin{definition}\textcolor{red}{(Open Set)}

A set $\Omega$ is called open if every point in it is an interior point.
\end{definition}

\begin{definition}\textcolor{red}{(Closed Set)}

A set $\Omega$ is called closed if its complement is an open set.
\end{definition}

\begin{definition}\textcolor{red}{(Connected Open Set)}

A open set $\Omega$ is called connected if it is not possible to find two disjoint open set $U_1$ and $U_2$ such that $\Omega=U_1\cup U_2$.
\end{definition}

\begin{definition}\textcolor{red}{(Region)}

A region is an open connected set.
\end{definition}

Now, we properly define the holomorphic here. 

\begin{definition} \label{def:holomorphic} \textcolor{red}{(Holomorphic Function)}

Let $f$ be a complex function on an open set $\Omega$. $f$ is holomorphic at the point $z\in\Omega$ if $\frac{f(z+h)-f(z)}{h}$ converges, as $h$ converges to $0$. Its limited is denoted as $f'(z)$. $f$ is called a holomorphic function if it is holomorphic at every point point in $\Omega$. 
\end{definition}

\begin{example}
$f(z)=z$ is holomorphic because $\frac{f(z+h)-f(z)}{h}=\frac{z+h-z}{h}=1$
\end{example}

\begin{example} \label{ex:non_holomorphic}
$f(z)=\bar{z}$ is not holomorphic because $\frac{f(z+h)-f(z)}{h}=\frac{\bar{z}+\bar{h}-\bar{z}}{h}=\frac{\bar{h}}{h}$. If $h$ is real, then $f'(z)=1$, but if $h$ is imaginary, $f'(z)=-1$.
\end{example}

It is clear from Definition \ref{def:holomorphic} that $f$ is holomorphic at a point $z\in\Omega$ if and only if there exists a complex number $a$ such that 

\begin{equation} \label{holomorphic}
f(z+h)-f(z)-ah=h\psi(h),
\end{equation}
where $\psi$ is a function defined for small $h$ and $\lim_{h\rightarrow0}\psi(h)=0$. \eqref{holomorphic} can be used to show the following properties.

\begin{proposition}
If $f$ and $g$ are holomorphic in $\Omega$, then
\begin{enumerate}
\item
$f+g$ is holomorphic

\item
$fg$ is holomorphic

\item
$(f+g)'=f'+g'$

\item
$(fg)'=f'g+fg'$
\end{enumerate}
\end{proposition}

The notion of complex differentiability differs significantly from the real differentiability. As seen from Example \ref{ex:non_holomorphic}, if we interpret the complex function $f(z)=\bar{z}$ as a real function with two coordinates, then $f(z)=f(x,-y)$, which is differentiable in the real sense. Now, we build the link between real and complex functions through the Cauchy-Riemann equation. 

Suppose $f$ is holomorphic at $z_0=x_0+iy_0$, and let $h=h_1+ih_2$. Consider $f(z)=f(x,y)$. If $h_1=0$, we have

\begin{equation*}
\lim_{h_2\rightarrow0}\frac{f(x_0,y_0+h_2)-f(x_0,y_0)}{ih_2}=\frac{1}{i}\frac{\partial f}{\partial y}.
\end{equation*}
Similarly, if $h_2=0$, we have

\begin{equation*}
\lim_{h_1\rightarrow0}\frac{f(x_0+h_1,y_0)-f(x_0,y_0)}{h_1}=\frac{\partial f}{\partial x}.
\end{equation*}
Since $f$ is holomorphic at $z_0$, we have the first Cauchy-Riemann equation as follows:

\begin{equation} \label{Cauchy_Riemann}
\frac{\partial f}{\partial x}=\frac{1}{i}\frac{\partial f}{\partial y}.
\end{equation}

If we write the holomorphic function as $f(z)=u(x,y)+iv(x,y)$, then we have the following relation:

\begin{equation} \label{Cauchy_Riemann_2}
\begin{aligned}
&\frac{\partial u}{\partial x}=\frac{\partial v}{\partial y}\\
&\frac{\partial v}{\partial x}=-\frac{\partial u}{\partial y}.
\end{aligned}
\end{equation}

We also have the following results from the Cauchy-Riemann equation

\begin{proposition}
If $f$ is holomorphic at $z_0$, then 

\begin{equation}
\frac{\partial f}{\partial\bar{z}}(z_0)=0 \quad \text{and} \quad f'(z_0)=\frac{\partial f}{\partial z}(z_0)=2\frac{\partial u}{\partial z}(z_0)
\end{equation}

\end{proposition}

\begin{proof}
For $z=x+iy$, we know $x=\frac{1}{2}(z+\bar{z})$ and $y=\frac{1}{2i}(z-\bar{z})$. Then, we can define the following two differential operators

\begin{equation*}
\begin{aligned}
&\frac{\partial f}{\partial z}=\frac{\partial f}{\partial x}\frac{\partial x}{\partial z}+\frac{\partial f}{\partial y}\frac{\partial y}{\partial z}=\frac{1}{2}\frac{\partial f}{\partial x}+\frac{1}{2i}\frac{\partial f}{\partial y}\\
&\frac{\partial f}{\partial \bar{z}}=\frac{\partial f}{\partial x}\frac{\partial x}{\partial \bar{z}}+\frac{\partial f}{\partial y}\frac{\partial y}{\partial \bar{z}}=\frac{1}{2}\frac{\partial f}{\partial x}-\frac{1}{2i}\frac{\partial f}{\partial y}.
\end{aligned}
\end{equation*}
From $\eqref{Cauchy_Riemann}$, we have the desired result $\frac{\partial f}{\partial\bar{z}}(z_0)=0$.

For the second result, first observe

\begin{equation*}
\frac{\partial f}{\partial z}(z_0)=\frac{1}{2}\frac{\partial f}{\partial x}(z_0)+\frac{1}{2i}\frac{\partial f}{\partial y}(z_0)=\frac{\partial f}{\partial x}(z_0)
\end{equation*}
Then, using the operator $\frac{\partial}{\partial z}$ defined above, we have
\begin{equation*}
\frac{\partial u}{\partial z}(z_0)=\frac{1}{2}\frac{\partial u}{\partial x}(z_0)+\frac{1}{2i}\frac{\partial u}{\partial y}(z_0)=\frac{1}{2}(\frac{\partial u}{\partial x}(z_0)+i\frac{\partial v}{\partial x}(z_0))=\frac{1}{2}\frac{\partial f}{\partial x}(z_0),
\end{equation*}
which directly leads to the second result.
\end{proof}

So far, we have assumed $f$ to be holomorphic and deduced the relation between its real and imaginary parts. In the next theorem, we show that the converse is also true, which completes the circle. 

\begin{theorem}
Suppose the complex function $f=u(x,y)+iv(x,y)$ is defined on an open set $\Omega$. If $u$ and $v$ are differentiable and satisfy the Cauchy-Riemann equation \eqref{Cauchy_Riemann_2} on $\Omega$, then $f$ is holomorphic on $\Omega$, and $f'(z)=\partial f/\partial z$.
\end{theorem}

\begin{proof}
Since $u$ and $v$ are differentiable, we have 

\begin{equation*}
\begin{aligned}
&u(x+h_1,y+h_2)-u(x,y)=\frac{\partial u}{\partial x}h_1+\frac{\partial u}{\partial y}h_2+|h|\psi_1(h)\\
&v(x+h_1,y+h_2)-v(x,y)=\frac{\partial v}{\partial x}h_1+\frac{\partial v}{\partial y}h_2+|h|\psi_2(h),
\end{aligned}
\end{equation*}
where $\psi_1(h),\psi_2(h)\rightarrow0$ as $|h|\rightarrow0$, and $h=h_1+ih_2$. Using \eqref{Cauchy_Riemann_2}, we have:

\begin{equation*}
\begin{aligned}
f(z+h)-f(z)&=f(x+f_1+iy+ih_2)-f(x+iy)\\
&=u(x+h_1,y+h_2)+iv(x+h_1,y+h_2)-u(x,y)-iv(x,y)\\
&=\frac{\partial u}{\partial x}h_1+\frac{\partial u}{\partial y}h_2+|h|\psi_1(h)+\frac{\partial v}{\partial x}h_1+\frac{\partial v}{\partial y}h_2+|h|\psi_2(h)\\
&=\left(\frac{\partial u}{\partial x}-i\frac{\partial u}{\partial y}\right)(h_1+ih_2)+|h|\psi_1(h)+|h|\psi_2(h),
\end{aligned}
\end{equation*}
which implies

\begin{equation*}
\begin{aligned}
\frac{f(z+h)-f(z)}{h}&=\left(\frac{\partial u}{\partial x}-i\frac{\partial u}{\partial y}\right)+\frac{|h|\psi_1(h)+|h|\psi_2(h)}{h}\\
&\rightarrow\left(\frac{\partial u}{\partial x}-i\frac{\partial u}{\partial y}\right)\\
&=2\frac{\partial u}{\partial z}\\
&=\frac{\partial f}{\partial z}.
\end{aligned}
\end{equation*}
\end{proof}

\begin{definition} \textcolor{red}{(Power Series)}

A power series is an expansion of the form $\sum^\infty_{n=0}a_nz^n$.
\end{definition}

\begin{definition} \textcolor{red}{(Absolute Convergence)}

A power series is said to converge absolutely if $\sum^\infty_{n=0}|a_nz^n|$ converges.
\end{definition}

\begin{definition} \textcolor{red}{(Analytic Function)}

We say that a function is analytic in an open set $\Omega$ if it has a convergent power series expansion in $\Omega$
\end{definition}

We give the following result about the convergence of a power series without proof. 

\begin{theorem} \textcolor{red}{(Convergence of Power Series)}

Given a power series $\sum^\infty_{n=0}a_nz^n$, there exists $0\leq R\leq\infty$ such that:

\begin{enumerate}
\item
If $|z|<R$, the series converges absolutely.

\item
If $|z|>R$, the series diverges.
\end{enumerate}

The radius of convergence $R$ is given by the Hadmard's formula as follows:

\begin{equation}
\frac{1}{R}=\limsup_{n\rightarrow\infty}|a_n|^{1/n}.
\end{equation}
\end{theorem}

Power series provide a very important class of analytic functions that are particularly simple to manipulate, as shown in the following theorem. 

\begin{theorem} \label{thm:power_series_conv} 
The power series $f(z)=\sum^\infty_{n=0}a_nz^n$ define a holomorphic function in its disk of convergence. The derivative of $f$ is also a power series obtained by differentiating term be term, i.e. $f'(z)=\sum^\infty_{n=0}na_nz^{n-1}$. Additionally, $f'(z)$ has the same radius of convergence. 
\end{theorem}

\begin{proof}
We aim to show the following:

\begin{equation*} 
\lim_{h\rightarrow0}|\frac{f(z+h)-f(z)}{h}-g(z)|=0,
\end{equation*}
where $g(z):=\sum^\infty_{n=0}na_nz^{n-1}$. Only main steps are shown here. The intuition is to separate the infinite sum of $f$ into a finite sum and a tail part, because the finite sum is essentially a polynomial, and its derivative is obtained by differentiating term by term. 

Define the following terms

\begin{equation*}
\begin{aligned}
&S_N:=\sum^N_{n=0}a_nz^n\\
&E_N:=\sum^\infty_{n=N+1}a_nz^n\\
&S'_N:=\sum^N_{n=0}na_nz^{n-1}.
\end{aligned}
\end{equation*}
Then, 
\begin{equation*}
\begin{aligned}
\lim_{h\rightarrow0}|\frac{f(z+h)-f(z)}{h}-g(z)|&=\lim_{h\rightarrow0}|\frac{S_N(z+h)-S_N(z)}{h}+\frac{E_N(z+h)-E_N(z)}{h}+S'_N-S'_N-g(z)|\\
&\leq \lim_{h\rightarrow0}\{|\frac{S_N(z+h)-S_N(z)}{h}-S'_N|+|\frac{E_N(z+h)-E_N(z)}{h}|+|S'_N-g(z)|\}.
\end{aligned}
\end{equation*}
The first term vanishes because polynomials are differentiable, and the derivate is obtained term by term. The last term vanishes because $\lim_{N\rightarrow\infty}S'_N=g(z)$. The second term also vanishes based on the following two facts

\begin{enumerate}
\item
$a^n-b^n=(a-b)(a^{n-1}+a^{n-2}b+a^{n-3}b^2+...+ab^{n-1}+b^{n-1})$

\item
$\sum^\infty_{n=N+1}na_nz^{n-1}$ is the tail of a convergence sequence and thus vanishes.
\end{enumerate}

Finally, $f'$ and $f$ have the same radius of convergence from the Hadmard's formula and the fact that $\lim_{n\rightarrow\infty}n^{1/n}=1$.

\end{proof}

Now, we change our focus to integration of complex functions along curves. First, we give several definitions

\begin{definition} \textcolor{red}{(Primitive)}

A primitive of $f$ on $\Omega$ is a function $F$ that is holomorphic on $\Omega$ and such that $F'(z)=f(z)$ for all $z\in\Omega$.
\end{definition}

\begin{definition} \textcolor{red}{(Parametrized Curve)}

A parametrized curve is a function $z(t)$ which maps a closed interval $[a,b]\subset\mathbb{R}$ to the complex plane.
\end{definition}

The integral of $f$ along a curve $\gamma$ is then defined as

\begin{equation*}
\int_\gamma f(z)dz:=\int_a^b f(z(t))z'(t)dt.
\end{equation*}

In order for this definition to make sense, one can also show that it is independent of the choice of the parametrization. By definition, the length of the smooth curve $\gamma$ is 

\begin{equation*}
\text{length($\gamma$)}=\int_a^b |z'(t)|dt.
\end{equation*}

We may then develop an useful bound, called the M-L bound, for the integration over curves:

\begin{equation} \label{M_L_bound}
\begin{aligned}
|\int_\gamma f(z)dz| = |\int^b_a f(z(t))z'(t)dt|\leq\int^b_a |f(z(t))z'(t)|dt\leq\sup_{z\in\gamma}|f(z(t))|\int^b_a |z'(t)|dt=\sup_{z\in\gamma}|f(z(t))|\text{length($\gamma$)}.
\end{aligned}
\end{equation}

Assuming $f$ has a primitive, many useful results arise.

\begin{theorem} \label{thm_int_end_points}
If $f$ has a primitive $F$ in $\Omega$, and $\gamma$ is a curve in $\Omega$ that has end points $w_1$ and $w_2$, then 

\begin{equation*}
\int_\gamma f(z)dz=F(w_1)-F(w_2)
\end{equation*}
\end{theorem}

\begin{proof}
The proof is quite straightforward. By definition,

\begin{equation*}
\begin{aligned}
\int_\gamma f(z)dz=\int^b_a f(z(t))z'(t)dt=\int^b_aF'(z(t))z'(t)dt=\int^b_a\frac{d}{dt}F(z(t))dt=F(z(b))-F(z(a))=F(w_1)-F(w_2).
\end{aligned}
\end{equation*}
\end{proof}

Next, we state several corollaries from this theorem, the proofs of which are very simple, so they are omitted. 

\begin{corollary} \label{cor:primitive_int_zero}
If $\gamma$ is a closed curve in an open set $\Omega$, and $f$ has a primitive in $\Omega$, then

\begin{equation} \label{primitive_int_zero}
\int_\gamma f(z)dz=0.
\end{equation}
\end{corollary}

\begin{example} \label{ex:1/z_no_primitive}
We will see that $f(z)=\frac{1}{z}$ does not have a primitive using the previous corollary. Parametrize an unit circle $\gamma$ centered at the origin by $z(\theta)=e^{i\theta}$, $dz=ie^{i\theta}$.

\begin{equation*}
\int_\gamma\frac{1}{z}dz=\int^{2\pi}_0\frac{1}{e^{i\theta}}ie^{i\theta}d\theta=2\pi i\neq0.
\end{equation*}
\end{example}

\begin{corollary}
If $f$ is holomorphic in a region $\Omega$, and $f'=0$, then $f$ is constant. 
\end{corollary}

\begin{corollary}
If $f$ has a primitive in $\Omega$, then the integral does not depend on the choice of the path. 
\end{corollary}

Corollary \ref{cor:primitive_int_zero} gives us one direction, and we would like to show the converse, namely if we know that \eqref{primitive_int_zero} for some types of curves $\gamma$, then a primitive for $f$ exists. We start with the Goursat's theorem.

\begin{theorem} \textcolor{red}{(Goursat)}

If $f$ is holomorphic in an open set $\Omega$, and $T\subset\Omega$ is a triangle whose interior is also contained in $\Omega$, then

\begin{equation*}
\int_T f(z)dz=0.
\end{equation*}

\end{theorem}

\begin{figure}
\centering
\includegraphics[scale=0.3]{Goursat.png} \\
\caption{Bisecting triangle in the proof of Goursat's Theorem.} 
\label{fig:goursat}
\end{figure}

\begin{proof}
The proof is based on this brilliant idea of recursively bisecting the triangle, as shown in Fig. \ref{fig:goursat}. The subscript counts the number of triangles, and the superscript represents the iteration. Only main steps of the proof are shown here. 

In the first iteration, we have:

\begin{equation*}
\begin{aligned}
\int_{T^{(0)}} f(z)dz&=\int_{T_1^{(1)}} f(z)dz+\int_{T_1^{(2)}} f(z)dz+\int_{T_1^{(3)}} f(z)dz+\int_{T_1^{(4)}} f(z)dz\\
&\leq4\int_{T_j^{(1)}} f(z)dz,
\end{aligned}
\end{equation*}
for some $j$ that gives the maximum integral value, and we name this triangle $T^{(1)}$. Repeating this process for $k$ times, we obtain

\begin{equation*}
\begin{aligned}
|\int_{T^{(0)}} f(z)dz|\leq4^k|\int_{T^{(k)}} f(z)dz|.
\end{aligned}
\end{equation*}

We may prove that there is a unique $z_0$ in every triangle $\{T^{(n)}\}_{n=1,...,k}$. Since $f$ is holomorphic at $z_0$, we have

\begin{equation*}
f(z)=f(z_0)+f'(z_0)(z-z_0)+\psi(z-z_0)(z-z_0),
\end{equation*}
where $\psi(z-z_0)\rightarrow0$ as $z\rightarrow z_0$. The constant $f(z_0)$ and $f'(z_0)(z-z_0)$ has primitives, so they have zero integral value over the closed curve $T^{(k)}$. Finally, $|\int_{T^{(k)}} f(z)dz|$ can be bounded by the M-L bound \eqref{M_L_bound}  and the following two facts:

\begin{enumerate}
\item
perimeter of $T^{(k)}$=$\frac{1}{2^k}\times$ perimeter of $T^{(0)}$

\item
diameter of $T^{(k)}$=$\frac{1}{2^k}\times$ diameter of $T^{(0)}$
\end{enumerate}

Then, it is straightforward to show $\lim_{k\rightarrow\infty}4^k|\int_{T^{(k)}} f(z)dz|=0$, which completes the proof. 

\end{proof}

Next, we prove the existence of primitive in a disk as a consequence of Goursat's Theorem. 

\begin{theorem} \label{thm:local_prim} \textcolor{red}{(Local Existence of Primitives)}

A holomorphic function in an open disk has a primitive in that disk.
\end{theorem}

\begin{figure}
\centering
\includegraphics[scale=0.3]{local_primitive_1.png} \\
\caption{Open disk with the choice of curve $\gamma_z$} 
\label{fig:local_primitive_1}
\end{figure}

\begin{figure}
\centering
\includegraphics[scale=0.3]{local_primitive_2.png} \\
\caption{Relation between $\gamma_z$ and $\gamma_{z+h}$.} 
\label{fig:local_primitive_2}
\end{figure}

\begin{proof}
We only show main steps. Consider the open disk $D$ centered at the origin, and choose a point $z\in D$ at random. Constructing the curve $\gamma_z$ as shown in Fig \ref{fig:local_primitive_1}. Define

\begin{equation*}
F(z):=\int_{\gamma_z}f(w)dw.
\end{equation*}
We claim that $F(z)$ is holomorphic in the open disk $D$ and $F'(z)=f(z)$, which means that $F$ is the primitive of $f$ in $D$. Now choose another point $z+h\in D$. Following the steps shown in Fig \ref{fig:local_primitive_2}, which uses the Goursat's Theorem, we have

\begin{equation*}
F(z+h)-F(z)=\int_{\gamma_{z+h}}f(w)dw-\int_{\gamma_{z}}f(w)dw=\int_\eta f(w)dw,
\end{equation*}
where $\eta$ is the curve shown in Fig. \ref{fig:local_primitive_2}(d). Since $f$ is holomorphic in $D$, it is continuous at $w$. Then, $f(w)=f(z)+\psi(w),$ where $\psi(w)\rightarrow0$ as $z\rightarrow w$. With the fact that $f(z)$ has a primitive and Theorem \ref{thm_int_end_points}, we have

\begin{equation*}
\lim_{h\rightarrow0}\frac{F(z+h)-F(z)}{h}=f(z),
\end{equation*}
which completes the proof.
\end{proof}

This theorem says that locally, every holomorphic function has a primitive. Now, we can use the previous result to get 

\begin{theorem} \label{thm:cauchy} \textcolor{red}{(Cauchy's Theorem for a Disk)}

If $f$ is a holomorphic function in a disk, then 

\begin{equation}
\int_\gamma f(z)dz=0,
\end{equation}
for every closed curve $\gamma$ in that disk.

\end{theorem}

\begin{proof}
Since $f$ is a holomorphic function in a disk, by previous result, $f$ has a primitive in that disk. Then, by Corollary \ref{cor:primitive_int_zero}, we have the desired result.
\end{proof}

\begin{corollary}
If $f$ is holomorphic in an open set that contains a circle $C$ and its interior, then

\begin{equation*}
\int_C f(w)dw=0.
\end{equation*}
\end{corollary}

\begin{proof}
We can slightly enlarge the disk with boundary circle $C$ such that $f$ is still holomorphic in it. Then, we can apply Cauchy's theorem to conclude the proof. 
\end{proof}

Cauchy's theorem also works for other shapes of contour, based on the same argument for the disk as shown in Theorem \ref{thm:cauchy}. Some examples of the toy contour is shown in Fig. \ref{fig:toy_contour}. The important point is that we can define without ambiguity the interior of a contour and construct the polygonal paths in an neighborhood of that contour and its interior, so that the primitive of the holomorphic function $f$ can be defined unambiguously (based on the Goursat's Theorem). For more detail, see the discussion on page 39-41 in the book. We will use the keyhole contour to prove the following theorem. Loosely speaking, if we know the value of a function on the boundary of a circle, we know its value at every point inside the circle.

\begin{figure}
\centering
\includegraphics[scale=0.3]{toy_contour.png} \\
\caption{Examples of toy contours.} 
\label{fig:toy_contour}
\end{figure}

\begin{theorem} \label{thm:cauchy_int} \textcolor{red}{(Cauchy's Integral Formula)}
Suppose $f$ is a holomorphic function in an open set that contains the closure of a disk $D$. If $C$ is the boundary circle of this disk with the positive orientation, then

\begin{equation*}
f(z)=\frac{1}{2\pi i}\int_C\frac{f(\zeta)}{\zeta-z}d\zeta,
\end{equation*}
for every point $z\in D$.
\end{theorem}

\begin{figure}
\centering
\includegraphics[scale=0.3]{keyhole.png} \\
\caption{Keyhole contour $\Gamma_{\delta,\epsilon}$.} 
\label{fig:keyhole_contour}
\end{figure}

\begin{proof}
We only show main steps. The proof is based on the keyhole contour $\Gamma$ (we omitted the subscript, since it is not relevant in this proof) in Fig. \ref{fig:keyhole_contour}. Since $f$ and $\frac{1}{\zeta-z}$ are holomorphic functions, we know $\frac{f(\zeta)}{\zeta-z}$ is holomorphic. Then, we have

\begin{equation*}
\int_\Gamma \frac{f(\zeta)}{\zeta-z}d\zeta=\int_C \frac{f(\zeta)}{\zeta-z}d\zeta + \int _{C_\epsilon} \frac{f(\zeta)}{\zeta-z}d\zeta=0,
\end{equation*}
where $C_\epsilon$ represents the small circle with radius $\epsilon$ in Fig. \ref{fig:keyhole_contour}. We can parametrize $C_\epsilon$ as follows:

\begin{equation*}
\zeta(\theta) = z+\epsilon e^{i\theta}, \quad d\zeta = i\epsilon e^{i\theta}.
\end{equation*}

Then, it is straightforward to get the result.

\end{proof}

As a corollary, we obtain further integral formulas expressing the derivatives of $f$ inside the disk in terms of the values of $f$ on the boundary.

\begin{corollary}
If $f$ is holomorphic in an open set $\Omega$, then $f$ has infinitely many complex derivatives in $\Omega$. Moreover, if $C\subset\Omega$ is a circle whose interior is also contained in $\Omega$, then 

\begin{equation*}
f^{(n)}(z)=\frac{n!}{2\pi i}\int_C\frac{f(\zeta)}{(\zeta-z)^{n+1}}d\zeta.
\end{equation*}
\end{corollary}

\begin{proof}
The proof is based on induction. Clearly, the formula holds for $n=0$. Suppose it holds for $n-1$, then we can show

\begin{equation*}
f^{(n)}(z)=\lim_{h\rightarrow0}\frac{f^{(n-1)}(z+h)-f^{(n-1)}(z)}{h},
\end{equation*}
by using the following fact: $a^n-b^n=(a-b)(a^{n-1}+a^{n-2}b+a^{n-3}b^2+...+ab^{n-1}+b^{n-1})$. We omitted the following computation.
\end{proof}

We have proved in Theorem \ref{thm:power_series_conv} that a power series is holomorphic in the interior of the disk of its convergence, now we show the converse.

\begin{theorem} \label{thm:holomorphic_power_series}
Suppose $f$ is holomorphic in an open set $\Omega$ that contains a disk $D$ centered at $z_0$ and its interior. Then, $f$ has a power series expansion at $z_0$

\begin{equation*}
f(z)=\sum_{n=0}^\infty a_n(z-z_0)^n, \quad \forall z\in D,
\end{equation*}
where
\begin{equation*}
a_n=\frac{f^{(n)}(z_0)}{n!}.
\end{equation*}
\end{theorem}

\begin{proof}
From Cauchy's integral formula, 

\begin{equation*}
\begin{aligned}
f(z)=\frac{1}{2\pi i}\int_C\frac{f(\zeta)}{\zeta-z}d\zeta&=\frac{1}{2\pi i}\int_C\frac{f(\zeta)}{\zeta-z+z_0-z_0}d\zeta \\
&=\frac{1}{2\pi i}\int_C\frac{f(\zeta)}{(\zeta-z_0)(1-\frac{z-z_0}{\zeta-z_0})}d\zeta.
\end{aligned}
\end{equation*}
Note $|\frac{z-z_0}{\zeta-z_0}|<1$, and we can apply geometric series expansion to $\frac{1}{1-\frac{z-z_0}{\zeta-z_0}}$ as 

\begin{equation*}
\begin{aligned}
\frac{1}{1-\frac{z-z_0}{\zeta-z_0}} = \sum_{n=0}^\infty (\frac{z-z_0}{\zeta-z_0})^n.
\end{aligned}
\end{equation*}

The result then easily follows.
\end{proof}

Next, we show the analytic continuation result, which was mentioned as one of the motivating points at the very beginning. Roughly speaking, a holomorphic function is determined if we know its value in an appropriate arbitrarily small subset. 

\begin{theorem}
Suppose $f$ is a holomorphic function in a region (connected and open set) $\Omega$ that vanishes on a sequence of distinct points with a limit point in $\Omega$, then $f$ is identically $0$.
\end{theorem}

\begin{proof}
Let the function $f$ vanishes at a sequence of points $\{w_n\}_{n=0}^\infty$, and the sequence converges to $z_0\in\Omega$. We first show that $f$ vanishes on all point in a small disk centered at $z_0$. From Theorem \ref{thm:holomorphic_power_series}, 

\begin{equation*}
f(z)=\sum_{n=0}^\infty a_n(z-z_0)^n,
\end{equation*}
for all $z$ in the small disk. Suppose that $f(z)$ is not identically $0$, then let $a_m$ be the first non-zero coefficient in the power series. Then, we have

\begin{equation*}
f(z)=a_m(z-z_0)^m+\sum_{n=m+1}^\infty a_n(z-z_0)^n=a_m(z-z_0)^m(1+g(z-z_0)),
\end{equation*}
where $g\rightarrow0$ as $z\rightarrow z_0$. Now take a vanishing point $w_k\neq z_0$ in that disk, we have 

\begin{equation*}
f(w_k)=a_m(w_k-z_0)^m(1+g(w_k-z_0))\neq0,
\end{equation*}
but this raises a contradiction, because $f(w_k)=0$ by the choice of $w_k$. We just proved that the value of $f$ on a small disk around $z_0$ is identically 0.

Now, we spread this disk to the entire $\Omega$. Define $U:=\text{int}\{z\in\Omega:f(z)=0\}$. $U$ is open by definition. If $\{z_n\}_{n=0}^\infty$ is a sequence of point in $U$ with limit $z$, then $f(z_n)=0$. By continuity, $f(z)=0$, implying that $z\in U$. It shows that $U$ is also closed. Now, define a set $V$ to be the complement of $U$. We have $U$ and $V$ are disjoint open sets that separate $\Omega$. Since $\Omega$ is connected, either $U$ or $V$ has to be empty. We showed in the first part of this proof that $z_0\in U$, so $V$ has to be empty. We have $U=\Omega$, which completes the proof. 

\end{proof}

\begin{corollary}
If $f$ and $g$ are holomorphic in a region $\Omega$, and $f(z)=g(z)$ for a sequence of distinct points which has the limit also in $\Omega$, then $f$ and $g$ agree on all $\Omega$.
\end{corollary}

\begin{proof}
It is a direct consequence of the previous theorem.
\end{proof}

Next, we consider singularities, in particular the different kind of point singularities that a holomorphic function can have. In order of increasing severity, these are:

\begin{enumerate}
\item
removable singularities

\item
poles

\item
essential singularities
\end{enumerate}

\begin{definition}\textcolor{red}{(Point Singularity/Isolated Singularity)}

A point singularity (or isolated singularity) of a function $f$ is a complex number $z_0$ such that $f$ is defined in a neighborhood of $z_0$ but not at the point $z_0$ itself.
\end{definition}

\begin{definition} \textcolor{red}{(Zeros)}

A function has a zero at $z_0$ if $f(z_0)=0$. Additionally, $z_0$ is called the isolated zero if $f(z)\neq0$ in the neighborhood of $z_0$.
\end{definition}

We use the following theorem to introduce the order of zeros for a holomorphic function. 

\begin{theorem} \label{thm:zero}
Suppose $f$ is a holomorphic function in a connected open set $\Omega$, and $f$ has a zeros at  a point $z_0\in\Omega$ but does not vanish identically in $\Omega$. Then, there exists a neighborhood $U\subset\Omega$ of $z_0$, a non-vanishing holomorphic function $g$ on $U$, and a unique positive integer $n$ such that 

\begin{equation}
f(z)=(z-z_0)^ng(z), \quad \forall z\in U.
\end{equation}
This unique integer $n$ is denoted as the order of zeros for $f$.
\end{theorem}

\begin{proof}
Since $\Omega$ is open and connected, and $f$ does not vanish identically on $\Omega$, we know that there is a neighborhood $U\subset\Omega$ of $z_0$ that $f$ is not identically zero in $U$. Since $f$ is holomorphic in $\Omega$, it has a power series expansion at $z_0$, and there is a smallest integer $n$ such that $a_n\neq0$. Then we can write

\begin{equation*}
f(z)=(z-z_0)^n[a^n+a_{n+1}(z-z_0)+...]=(z-z_0)^ng(z),
\end{equation*}
where $g(z)$ is holomorphic and nowhere vanishing since $a_n\neq0$. This integer $n$ is also unique. Suppose otherwise, $f(z)=(z-z_0)^ng(z)=(z-z_0)^mh(z)$ and $m>n$. Then, $g(z)=(z-z_0)^{m-n}h(z)$. As $z\rightarrow z_0$, we have $g(z)=0$, which is a contradiction. Same reasoning applies for the case of $m<n$.
\end{proof}

The importance of this result is that we can precisely describe the type of singularity possessed by the function $1/f$ at $z_0$.

\begin{definition} \textcolor{red}{(Deleted Neighborhood)}

The deleted neighborhood of $z_0$ is defined as an open disk centered at $z_0$ minus the point $z_0$. Equivalently, $\{z:0<|z-z_0|<r\}$.
\end{definition}

\begin{definition} \textcolor{red}{(Poles)}

A function $f$ defined in a deleted neighborhood of $z_0$ is said to have a pole at $z_0$ if the function $1/f$, defined to be zero at $z_0$, is holomorphic in a full neighborhood of $z_0$.
\end{definition}

We have the following theorem, as a direct result from the definition of poles and Theorem \ref{thm:zero}.

\begin{theorem}
If $f$ has a pole at $z_0\in\Omega$, then in a neighborhood of that point, there exist a non-vanishing holomorphic function $h$ and a unique positive integer $n$ such that 

\begin{equation}
f(z)=(z-z_0)^{-n}h(z).
\end{equation}
The integer $n$ is called the order of the pole. 
\end{theorem}

Since $h$ also has a power series expansion, we may write $f$ as follows

\begin{equation} \label{f_pole}
f(z)=\frac{a_{-n}}{(z-z_0)^n}+\frac{a_{-n+1}}{(z-z_0)^{n-1}}+...+\frac{a_{-1}}{(z-z_0)}+G(z),
\end{equation}
where $G(z)$ is a holomorphic function in the neighborhood of $z_0$. In this expression, the summation without the function $G(z)$ is called the principal part of $f$ at the pole $z_0$, and the coefficient $a_{-1}$ is called the residue of $f$ at that pole. All other terms in the principal part with order strictly greater than $1$ has primitives in a deleted neighborhood of $z_0$. Therefore if $P(z)$ denotes the principal part and $C$ is any circle centered at $z_0$, we have 

\begin{equation}
\frac{1}{2\pi i}\int_C P(z)dz=a_{-1},
\end{equation}

Based on \eqref{f_pole}, we can compute the residue by

\begin{equation}
\text{res}_{z_0}f=\lim_{z\rightarrow z_0}\frac{1}{(n-1)!}(\frac{d}{dz})^{n-1}(z-z_0)^nf(z).
\end{equation}
If $f=\frac{n(z)}{d(z)}$ has a simple pole at $z_0$, using the L'Hopitole's rule, we have a simpler formula to calculate residue

\begin{equation}
\text{res}_{z_0}f=\frac{n(z_0)}{d'(z_0)}.
\end{equation}

Then, we give the celebrated residue formula.

\begin{theorem} \label{thm:res_formula} \textcolor{red}{(Residue Formula)}

Suppose that $f$ is holomorphic in an open set containing a circle $C$ and its interior, except for poles at the points $z_1,...,z_N$ inside $C$. Then,

\begin{equation}
\int_Cf(z)dz=2\pi i \sum^N_{k=1}\text{res}_{z_k}f.
\end{equation}
\end{theorem}

\begin{proof}
We only prove the case when $f$ has a simple pole at $z_0$. The generalization to the finite number of poles is straightforward.

Consider the same keyhole toy contour as shown in Fig. \ref{fig:keyhole_contour}, where the small circle is centered at the pole $z_0$. As the width of the corridor goes to $0$, we have 

\begin{equation*}
\int_Cf(z)dz=\int_{C_\epsilon}f(z)dz.
\end{equation*}
Then, by the Cauchy's integral formula \ref{thm:cauchy_int} with a constant function $f(z)=a_{-1}$, we have 

\begin{equation*}
\frac{1}{2\pi i}\int_{C_\epsilon}\frac{a_{-1}}{z-z_0}dz=a_{-1}.
\end{equation*}
For the same reasoning, we have 
\begin{equation*}
\frac{1}{2\pi i}\int_{C_\epsilon}\frac{a_{-1}}{(z-z_0)^k}dz=0,
\end{equation*}
for all $k>1$. Referring to \eqref{f_pole}, we have 
\begin{equation*}
\int_Cf(z)dz=\int_{C_\epsilon}f(z)dz=2\pi ia_{-1}=2\pi i\text(res)_{z_0}f,
\end{equation*}
since $G(z)$ is holomorphic in the neighborhood of $z_0$.
\end{proof}

Now, we give a definition of another singularity. 

\begin{definition} \textcolor{red}{(Removable Singularity)}

Let $f$ be a function holomorphic in an open set $\Omega$ except possibly at one point $z_0$ in $\Omega$. If we can define $f$ at $z_0$ in such a way that $f$ becomes holomorphic in all of $\Omega$, we say that $z_0$ is a removable singularity for $f$.

\end{definition}

Here is an equivalent way of identifying a removable singularity, but we are not going to prove it. 

\begin{theorem}
$f$ has a removable singularity at $z_0$ if and only if $f$ is holomorphic in a deleted neighborhood $D$ of $z_0$ and $f$ is bounded in $D-z_0$.
\end{theorem}
Based on this theorem, we will also have an equivalent saying for the pole.

\begin{theorem}
Suppose that $f$ has an isolated singularity at the point $z_0$. Then $z_0$ is a pole of $f$ if and only if $|f(z)|\rightarrow\infty$ as $z\rightarrow z_0$.
\end{theorem}

\begin{proof}
If $z_0$ is a pole, then $1/f$ has a zero at $z_0$, therefore $|f(z)|\rightarrow\infty$ as $z\rightarrow z_0$. Conversely, if $|f(z)|\rightarrow\infty$ as $z\rightarrow z_0$, then $1/|f(z)|\rightarrow 0$ as $z\rightarrow z_0$. So, by previous theorem, $1/f(z)$ has removable singularity at $z_0$ and vanish at that point. By definition, it means that $z_0$ is a pole of $f$.
\end{proof}

Let's make a summary here. Isolated singularities belong to one of the following three categories:

\begin{enumerate}
\item
Removable singularity ($f$ bounded near $z_0$)

\item
Pole singularity ($|f(z)|\rightarrow\infty$ as $z\rightarrow z_0$)

\item
Essential singularity
\end{enumerate}

By default, any singularity that is not removable or a pole is defined to be an essential singularity. Contrary to the controlled behavior of a holomorphic function near a removable singularity or a pole, it is typical for a holomorphic function to behave erratically near an essential singularity. For example, $e^{1/z}$ has an essential singularity at the origin. It grows to indefinitely as $z$ approaches $0$ on the positive real line, while it approaches $0$ as $z$ goes to $0$ on the negative real axis. Finally, it oscillates rapidly, yet remains bounded, as $z$ approaches the origin on the imaginary axis. 

Now, we turn to functions with only isolated singularities that are poles.

\begin{definition} \textcolor{red}{(Meromorphic Functions)}
A function on an open set $\Omega$ is meromorphic if there exists a sequence of points $\{z_1,z_2,...\}$ that have no limit points in $\Omega$, and such that 

\begin{enumerate}
\item
$f$ is holomorphic in $\Omega-\{z_1,z_2,...\}$.

\item
$f$ has poles at $\{z_1,z_2,...\}$.
\end{enumerate}
\end{definition}

We say that $f(z)$ has a pole singularity (removable singularity, essential singularity, respectively) at infinity if $f(1/z)$ has a pole singularity (removable singularity, essential singularity, respectively) at the origin. 

\begin{definition} \textcolor{red}{(Meromorphic in the Extended Complex Plane)}
A meromorphic function in the complex plane that is either holomorphic at infinity or has a pole at infinity is said to be  meromorphic in the extended complex plane.
\end{definition}

We won't prove the following theorem.

\begin{theorem}
The meromorphic function in the extended complex plane are the rational functions.
\end{theorem}

Next, we discuss the argument principle and applications. First notice that the function $\log f(z)$ is multi-valued. In face, $\log f(z)=\log|f(z)|+i\arg f(z)$, where $\arg f(z)$ causes the trouble because it is only determined unambiguously up to an additive integer multiple of $2\pi$. $\log f(z)$ has the aforementioned form because of the following reason. Let $f(z)=re^{i\theta}$, then

\begin{equation*}
f(z)=e^{\log f(z)}=e^{Re(\log f(z))+i Im(\log f(z)}=e^{Re(\log f(z))}e^{i Im(\log f(z)},
\end{equation*}
then, matching the formula, we have

\begin{equation*}
\begin{aligned}
&\log r=\log |f(z)|=Re(\log f(z)),\\
&\theta=Im(\log f(z)),
\end{aligned}
\end{equation*}
which gives us $\log f(z)=Re(\log f(z))+i Im(\log f(z))=\log |f(z)|+i \arg f(z)$.

Although $\log f(z)$ is multi-valued, its derivative $f'(z)/f(z)$ is single valued, and $\int_\gamma \frac{f'(z)}{f(z)}dz$ can be interpreted as the net change in the argument of $f(z)$ as $z$ traverses the curve $\gamma$. Then, we have the following theorem. 

\begin{theorem} \textcolor{red}{(Argument Principle)}
Suppose $f$ is a meromorphic function in an open set containing a circle $C$ and its interior. If $f$ has no poles and never vanishes on $C$, then

\begin{equation} 
\frac{1}{2\pi i}\int_C \frac{f'(z)}{f(z)}dz=\text{(number of zeros of $f$ inside $C$)}-\text{(number of poles of $f$ inside $C$)},
\end{equation}
where zeros and poles are counted with multiplicities.
\end{theorem}

\begin{proof}
Suppose $f$ has a zero of order $n$ at $z_0$, then we may write

\begin{equation*}
f(z)=(z-z_0)^ng(z), 
\end{equation*}
where $g(z)$ is holomorphic and non-vanishing in the neighborhood around $z_0$. It implies

\begin{equation*}
\frac{f'(z)}{f(z)}=\frac{n}{z-z_0}+\frac{g'(z)}{g(z)},
\end{equation*}
which means that $\frac{f'(z)}{f(z)}$ has a simple pole with residue $n$ at $z_0$. Similarly, we may show that if $f$ has a pole of order $n$ at $z_0$, then $\frac{f'(z)}{f(z)}$ has a simple pole at with residue $-n$ at $z_0$. In conclusion, since $f$ is meromorphic, the function $f'/f$ has simple poles at the zeros and poles of $f$, and the residue is simply the order of the zero of $f$ or the negative of the order of the pole of $f$. Using the residue formula \ref{thm:res_formula}, we have the desired result. 

Note that the assumption of the function being meromorphic comes in when we applied the residue formula, which requires the only singularities begin the poles. 

\end{proof}

An application of the argument principle is the Rouche's theorem, which says that holomorphic function can be perturbed slightly without changing the number of its zeros. 

\begin{theorem} \label{thm:rouche} \textcolor{red}{(Rouche)}

Suppose that $f$ and $g$ are holomorphic in an open set containing a circle $C$ and its interior. If 

\begin{equation*}
|f(z)|>|g(z)| \quad \forall z\in C,
\end{equation*}
then $f$ and $f+g$ have the same number of zeros inside the circle. 
\end{theorem}

\begin{figure}
\centering
\includegraphics[scale=0.3]{Rouche.png} \\
\caption{Proof of Rouche's theorem.} 
\label{fig:rouche}
\end{figure}


\begin{proof}
The idea behind the proof is illustrated in Fig. \ref{fig:rouche}. We can interpret the left hand side of the argument principle as the number of times that $f(z)$ encircles the origin as $z$ travels along the curve $\gamma$. We can see from Fig. \ref{fig:rouche} that by the setup of the theorem, $f+g$ can never include the origin. Thus, by the argument principle, $f$ and $f+g$ have the same number of zeros inside the circle.
\end{proof}

Now, we can state some geometric properties of holomorphic functions when they are considered as mappings. 

\begin{definition} \textcolor{red}{(Open Mapping)}

A mapping is said to be open if it maps open set to open set.
\end{definition}

\begin{theorem} \label{thm:open_map} \textcolor{red}{(Open Mapping)}

If $f$ is holomorphic and non-constant in a region $\Omega$, than $f$ is open.
\end{theorem}

\begin{proof}
Let $w_0$ belongs to the range of $f$, t.e., $f(z_0)=w_0)$. We need to show that the neighborhood of $w_0$ is also contained in the image of $f$. 

Define

\begin{equation*}
g(z):=f(z)-w=f(z)-w_0+w_0-w=:F(z)+G(z).
\end{equation*}

Find $\delta$ such that the set $\{z:|z-z_0|\leq \delta\}$ is contained in $\Omega$ and $f(z)\neq w_0$ on the circle $|z-z_0|=\delta$. Find $\epsilon$ such that $|f(z)-w_0|\geq\epsilon$ on the circle $|z-z_0|=\delta$. Then, for $z$ on the circle $|z-z_0|=\delta$, we have $|f(z)-w_0|\geq|w-w_0|$ if $|w-w_0|<\epsilon$. From Rouche's theorem \ref{thm:rouche}, we know that $g(z)$ has the same zeros as for $F(z)$. We know that $F(z)$ as a zero at $z_0$ by assumption, which implies $g(z)$ also has a zero. The proof is complete.

\end{proof}

The next result pertains to the size of a holomorphic function. We refer to the maximum of a holomorphic function in an open set $\Omega$ as the maximum of its absolute value IN $\Omega$.

\begin{theorem} \textcolor{red}{(Maximum Modulus Principle)}
If $f$ is a non-constant holomorphic function in a region $\Omega$, then $f$ cannot attain a maximum in $\Omega$.
\end{theorem}

\begin{proof}
Suppose $f$ attains a maximum at $z_0$. Since $f$ is a non-constant holomorphic function on $\Omega$, from the open mapping theorem \ref{thm:open_map}, $f$ is an open mapping. Then, there exists an open set $D\subset\Omega$ centered at $z_0$, such that its image $f(D)$ is open and contains $f(z_0)$. It implies that $z_0$ is not the maximum, which is a contradiction.
\end{proof}

\begin{corollary}
Suppose that $\Omega$ is a region with compact closure $\bar{\Omega}$. If $f$ is holomorphic in $\Omega$ and continuous in $\bar{\Omega}$, then

\begin{equation*}
\sup_{z\in\Omega}|f(z)|\leq\sup_{z\in\bar{\Omega}-\Omega}|f(z)|
\end{equation*}
\end{corollary}

Since $f$ is continuous in the compact set $\bar{\Omega}$, $|f|$ attains its maximum in $\bar{\Omega}$.

Now, we switch to find the general form of the Cauchy's theorem. The key is to understand in what regions we can define the primitive of a given holomorphic function. Then, by Corollary \ref{cor:primitive_int_zero}, we can conclude the Cauchy's theorem. This requires the notion of homotopy and the resulting idea of simple-connectivity. 

\begin{definition} \textcolor{red}{(Simply Connected)}

A region $\Omega$ in the complex plane is simply connected if any two pair of curves in $\Omega$ with the same end-points are homotopic. Loosely speaking, two curves are homotopic if one curve can be deformed into the other by a continuous transformation without ever leaving $\Omega$.
\end{definition}

We give the general form of the Cauchy's theorem without proving it. 

\begin{theorem}
If $f$ is holomorphic in the simply connected region $\Omega$, then 

\begin{equation*}
\int_\gamma f(z)dz=0,
\end{equation*}
for any closed curve $\gamma$ in $\Omega$.
\end{theorem}

We have mentioned previously that the function $\log f(z)$ is not single-valued. To make sense of the logarithm as a single-valued function, we must restrict the set on which we define it. This is so-called choice of a branch or sheet of the logarithm. We have the following global definition of a branch of the logarithm function based on simply connected domain. 

\begin{theorem}
Suppose $\Omega$ is simply connected with $1\in\Omega$ and $0\notin\Omega$. Then in $\Omega$ there is a branch of the logarithm $F(z)=\log_\Omega(z)$ so that 

\begin{enumerate}
\item
$F$ is holomorphic in $\Omega$.

\item
$e^{F(z)}=z$ for all $z\in\Omega$.

\item
$F(r)=log(r)$ whenever $r$ is a real number and near $1$
\end{enumerate}
\end{theorem}

\begin{proof}
We shall construct $F$ as a primitive of $1/z$. The rest of the steps are omitted.
\end{proof}

\begin{figure}
\centering
\includegraphics[scale=0.3]{principal_branch.png} \\
\caption{Path of integration for the principal branch of the logarithm.} 
\label{fig:principal_branch}
\end{figure}

In the slit plane $\Omega=\mathbb{C}-\{(-\infty,0]\}$, we have the principal branch of the logarithm $\log z=\log r+i\theta$, where $z=re^{i\theta}$ with $|\theta|<\pi$. Path of integration for the principal branch of the logarithm is shown in Fig. \ref{fig:principal_branch}. Then,

\begin{equation*}
\log z=\int^r_1\frac{1}{x}dx+\int^\theta_0\frac{1}{w}dw=\log r+i\theta.
\end{equation*}

Notice that in general 

\begin{equation*}
\log(z_1z_2)\neq\log z_1 +\log z_2.
\end{equation*}

Having defined a logarithm on a simply connected domain, we can take easily take an inter power of a complex number $z^n$ as $z^n=e^{n\log z}$. But what about the complex power? We may use the following theorem.

\begin{theorem}
If $f$ is a nowhere vanishing holomorphic function in a simply connected region $\Omega$, then there exists a holomorphic function $g$ on $\Omega$ such that 

\begin{equation*}
f(z)=e^{g(z)}.
\end{equation*}
\end{theorem}

\begin{proof}
Fix a point $z_0$ in $\Omega$, and define

\begin{equation*}
g(z):=\int_\gamma \frac{f'(w)}{f(w)}dw+c_0,
\end{equation*}
where $\gamma$ is the curve from $z_0$ to $z$, and $c_0$ is a constant such that $f(z_0)=e^{g(c_0)}$. This definition is independent of the path $\gamma$ since $\Omega$ is simply connected. Since $\frac{f'(w)}{f(w)}$ is holomorphic, we can prove that $g$ is holomorphic using the same technique as in Theorem \ref{thm:local_prim}, and 

\begin{equation*}
g'(z):=\frac{f'(z)}{f(z)}.
\end{equation*}
We also know that $f(z)e^{-g(z)}$ is a constant because 

\begin{equation*}
\frac{d}{dz}(f(z)e^{-g(z)})=f'(z)e^{-g(z)}-f(z)g'(z)e^{-g(z)}=0.
\end{equation*}
Additionally $f(z_0)e^{-g(z_0)}=1$ implies that $f(z)=e^{g(z)}$ for all $z\in\Omega$. The proof is complete. 


\end{proof}

























\end{document}